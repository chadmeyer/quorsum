\documentclass[\letterpaper]{article}

\begin{document}

\title{Quorsum endgame strategic analysis}
\author{Chad D. Meyer // Los Alamos National Laboratory}
\date{today}
\maketitle

\begin{abstract}
Quorsum is a racing game in which players attempt to move their pawns across
the board by rolling dice.  Additionally, pawns can only move on tiles of
their own colors.  Players can assign their limited dice to flipping tiles
and to moving their pawns toward their homes.  In this note, we discuss some
of the statistical and probablistic considerations surrounding late-game
decision making in a game of quorsum.
\end{abstract}

\section{Introduction}
Quorsum was introduced by Steve and Will Erickson in circa 2014.  Although
originally called an abstract strategy game, there is some debate whether the
random element of dice roling excludes quorsum from the category.  Nevertheless,
there is no hidden information, and players make their decisions based on the
board layout, current pawn positions and probablilities of rolling certain
numbers on their dice.

Recently, Erickson (2017) presented an analysis of the probabilities of
successfully moving $j$ spaces when starting with $n$ dice, which was analysed
using Markov theory to predict the expected number of turns to reach home.  This
has obvious implications to endgame strategy but is not enough to fully discuss
the possibilities.

This paper will be structured as follows.  In Secion 2, we outline the rules of
quorsum of relevance to our analysis.  In Secion 3, we 

\end{document}
